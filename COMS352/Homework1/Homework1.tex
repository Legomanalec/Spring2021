\title{COM S 352 Homework 1}
\author{Alec Meyer}

\date{\today}

\documentclass[11pt]{book}
\newcommand\tab[1][1cm]{\hspace*{#1}}
\usepackage{amssymb}
\usepackage{listings}

\begin{document}
\maketitle


\section*{Question 1}
Interrupts are used to inform a program or proccess
of an external proccess, typically a physical 
device controller telling the CPU that there has 
been an I/O operation. A Trap, However, is an interrupt 
triggered by software (i.e. exceptions, requests
from other programs). Yes, traps can be generated 
intentionally such as trying to access invalid memory
or arithmetic exceptions.
\section*{Question 2}
The two modes of a CPU are User mode and Kernel mode.
The system privlleges are increased when Kernal mode
is in use because Kernal mode will allow more control
over the system. This dual operation will help protect
the user and the system from programs not laucnhing 
correctly and/or being malicious. The reason to 
distinguish the two is to know where Interrupts
should be used. The system will produce interuppts 
in user mode then switch to Kernel mode. Kernel mode 
can only run privileged instructions.
\section*{Question 3}
Clear memory\\
turn off Interrupts
put CPU in Kernel mode
\section*{Question 4}

\section*{Question 5}

\section*{Question 6}

\section*{Question 7}
\end{document}

