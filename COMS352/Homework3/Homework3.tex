\title{COM S 352 Homework 3}
\author{Alec Meyer}

\date{\today}

\documentclass[11pt]{article}
\usepackage{changepage}
\usepackage{graphicx}
\usepackage{amsmath}
\graphicspath{ {./images/} }
\newcommand\tab[1][1cm]{\hspace*{#1}}
\usepackage{amssymb}


\begin{document}
\maketitle

\section*{Question 1}
Heap memory\\
Global variables

\section*{Question 2}
No, the operating system only sees a single process at
a time so it will not schedule other threads. Being able 
to have multiple user-level threads does not result in 
parallelism. That means the kernel can only schedule 
one kernel thread at a time. 

\section*{Question 3}
yes, generally concurrency occurs when you 
are working with a single core system.
Single core systems cannot have parallelism so the threads 
must be scheduled over time intervals, while parallelism allows
multiple threads to run simultaneously.

\newpage
\section*{Question 4}
$\frac{1}{S + \frac{1-S}{N}}$
or
$\frac{1}{1 - P + \frac{P}{N}}$\\\\
A.\\
$\frac{1}{0.05 + \frac{1-0.05}{8}} = 5.926$\\
$\frac{1}{0.05 + \frac{1-0.05}{16}} = 9.143$\\
$\frac{9.143}{5.926} = 1.543times$\\\\
B.\\
$\frac{1}{0.50 + \frac{1-0.50}{8}} = 1.778$\\
$\frac{1}{0.50 + \frac{1-0.50}{16}} = 1.882$\\
$\frac{1.882}{1.778} = 1.058times$\\\\
A system depends on a lot more than just number of 
processing cores when measuring speeds. Algorithm efficiency,
hardware efficiency also play a factor in speed.

\section*{Question 5}
A. You would need two threads total, one for input and 
one for output. There is no reason to add anymore threads, 
efficiency will not be affected.\\
B. Four threads, because we want the same amount of 
threads as cores for the system since each core can 
use one thread at a time. Any less threads and we are 
wasting computing, and more threads than cores wouldn't 
benefit the system.

\newpage
\section*{Question 6}
\begin{verbatim}
#include <pthread.h>
#include <stdio.h>
#include <unistd.h>
#include <sys/wait.h>
int id[4];
/* the thread */
void *runner(void *param) 
{
    int *id = (int *)param;
    printf("THREAD %d FINISHED\n", *id);
    pthread_exit(0);
}
int main(int argc, char *argv[]) 
{
    pid_t pid;
    pthread_t tid[4];
    pthread_attr_t attr;
    pid = fork();
    pthread_kill(tid[1], 3);

    if (pid == 0) 
    {
        /* child process */
        pthread_attr_init(&attr);
        for (int i=0; i<4; i++) 
        {
            id[i] = i;
            pthread_create(&tid[i], &attr, runner, &id[i]);
        }
        for (int i = 0; i < 4; i++)
            pthread_join(tid[i], NULL); //waits for each thread

        printf("CHILD PROCESS FINISHED\n");
    } 
    else if (pid > 0) 
    {
        /* parent process*/
        wait(NULL); //waits for children
        printf("PARENT PROCESS FINISHED\n");
    }
    return 0;
}


OUTPUT:
    THREAD 0 FINISHED
    THREAD 1 FINISHED
    THREAD 2 FINISHED
    THREAD 3 FINISHED
    CHILD PROCESS FINISHED
    PARENT PROCESS FINISHED
\end{verbatim}
\end{document}
