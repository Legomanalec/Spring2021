\title{COM S 352 Homework 9}
\author{Alec Meyer}

\date{\today}

\documentclass[11pt]{article}
\usepackage{changepage}
\usepackage{graphicx}
\usepackage{amsmath}
\graphicspath{ {./images/} }
\newcommand\tab[1][1cm]{\hspace*{#1}}
\usepackage{amssymb}


\begin{document}
\maketitle

\section*{Question 1}
With links to the deleted file still exsisting there could be 
a dangling pointer. a solution to this would be to add
a counter to the file which represents the number of references
to a shared file, then when you add a new link increment, if
you delete a link you decrement the counter, then the file can 
only be deleted when it has a count of 0.
\section*{Question 2}
    The operating system should maintain just one table that 
    contains references to the files that are currentky being 
    accessed by all users. If two users are creating 
    entries, then the operating system will need separate
    entries for each user. 
\section*{Question 3}
\begin{itemize}
    \item \textbf{Contiguous:}
        This technique can result in internal and external 
        fragmentation. A dynamic file size is also hard for 
        this technique because it needs to find an open block
        of length N and is not expecting a file to gain size. 
    \item \textbf{Linked:}
        Since this technique works like a linked list there is 
        no external fragmentation. This technique is only really 
        usable for sequential file access.
    \item \textbf{Indexed:}
        This technique has no issue with external fragmentation 
        because it just points to certain clumps of memory from 
        a central block of pointers. Only issue with this technique
        is that it does have to hold a lot of pointers which could 
        take up a lot of space.
\end{itemize}
\section*{Question 4}
\begin{itemize}
    \item \textbf{Contiguous:}
        1 region
    \item \textbf{Linked without a FAT:}
        4 blocks
    \item \textbf{Linked with a FAT:}
        4 blocks
    \item \textbf{Indexed:}
        4 blocks
\end{itemize}
\section*{Question 5}
\begin{itemize}
    \item \textbf{a.}
        Since memory will be utalized effectivly then we 
        dont have to worry about a small amount of data taking
        up an entire region. 
    \item \textbf{b.}
        We would just need to makre sure that when memory
        is deallocated that the spaces are not left. 
\end{itemize}

\section*{Question 6}
    16kb = 16384bytes\\
    $\frac{16384}{8}=2048$\\
    16 blocks\\
    $16*16 = 256$ direct\\
    $256 + 2048*16 = 33024$ single\\
    $33024 + 2048*2048*16 = 67141888$ double\\
    $67141888 + 2048*2048*2048*16 = 137506095360$ triple\\
    maximum size = 137 Gb


\end{document}