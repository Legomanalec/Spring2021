\title{COM S 352 Homework 5}
\author{Alec Meyer}

\date{\today}

\documentclass[11pt]{article}
\usepackage{changepage}
\usepackage{graphicx}
\usepackage{amsmath}
\graphicspath{ {./images/} }
\newcommand\tab[1][1cm]{\hspace*{#1}}
\usepackage{amssymb}


\begin{document}
\maketitle

\section*{Question 1}
Two bidders could call the bid function concurrently, which
results in the highestBid variable to be shared. The race condition
is that that two users could be on two separate threads and bid
the values 10 and 20. It is possible for the 10 bid to be used 
if highestBid is shared between the two. \\
\begin{verbatim}
    void bid(double amount) {
        acquire(highestbid)
        if (amount > highestBid)
        highestBid = amount;
    }
    release();
\end{verbatim}

\newpage
\section*{Question 2}
\begin{verbatim}
    typedef struct {
        int available;
    } lock;

    void acquire(lock *mutex) {
        while (mutex->available == 1); 
            mutex->available = 1; 
    }

    void release(lock *mutex) {
        mutex->available = 0;
    }

    void initialization_code(lock *mutex) {
        mutex->available = 0;
    }
\end{verbatim}


\section*{Question 3}
\textbf{1:} A spinlock would be a better locking mechanism
because it doesn't require a context switch and can avoid busy 
waiting. This makes them more preferable for short durations.\\
\textbf{2:} A mutex lock would be preferable for a long 
duration as it allows other processes to run while the lock
waits.\\
\textbf{3:} A mutex lock would be prefered for this scenario as it 
doesn't need to be spinning while waitng for the thread to wake up.

\newpage
\section*{Question 4}
\textbf{a.} The race condition appears in the variable 
\emph{number\_of\_processes} as the method 
\emph{allocate\_process()} and \emph{release\_process()} 
modify its value. \\\\
\textbf{b.}
\begin{verbatim}
    #define MAX_PROCESSES 255
    int number_of_processes = 0;
    /* the implementation of fork() calls this function */
    int allocate_process() {
        int new_pid;
        acquire(mutex);/////////////////////////////
        if (number_of_processes == MAX_PROCESSES) {
            release(mutex);/////////////////////////
            return -1;
        } else {
            /* allocate necessary process resources */
            ++number_of_processes;
            release(mutex);/////////////////////////
            return new_pid;
         }
    }
    /* the implementation of exit() calls this function */
    void release_process() {
        /* release process resources */
        acquire(mutex);/////////////////////////////
        --number_of_processes;
        release(mutex);/////////////////////////////
    }
\end{verbatim}


\section*{Question 5}
For monitors, a \emph{signal()} call will resume only one 
process if there is a waiting process. However, when there 
is not a waiting process the call to \emph{signal()} will 
not be saved.

\end{document}