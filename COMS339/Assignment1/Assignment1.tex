\title{COM S 339 Assignment 1}
\author{Alec Meyer}

\date{\today}

\documentclass[12pt]{article}
\usepackage{changepage}
\usepackage{graphicx}
\usepackage{amsmath}
\graphicspath{ {./images/} }
\newcommand\tab[1][1cm]{\hspace*{#1}}
\usepackage{amssymb}
\usepackage{setspace}
\usepackage[margin=1in]{geometry}

\doublespacing
\begin{document}
\maketitle


\section*{\tab  Emergence of Architecture Through Refactoring}
\tab Agile is one of the most common software development practices and is used
widely throughout the tech industry. Agile is also what we are talking about
when we discuss satisfactory architecture being built from refactoring. This
essay will discuss the downfalls of producing software through constant 
refactoring and failing a project’s needs. 

\tab Satisfactory architecture emerging from refactoring is a considerable debate 
in the software architecture community.  The idea is that a project can be 
developed without traditional architecture through refactoring and editing 
code. Agile development depends on this claim. This ideology allows engineers 
to disregard the future of the project’s architecture based on refactoring any 
future issues. 

\tab Satisfactory architectures can also be shown to not emerge from refactoring. 
”Failure [architecture did not emerge] is also more prevalent in cases of large
pre-existing products which either had no discernable architecture to begin 
with or had their architecture erode away over years of maintenance and 
haphazard addition of new features by persons who didn’t  understand the 
pre-existing architecture or ignored it for one reason or another.”  This 
quote from the paper is one example of how satisfactory architecture does 
not arise from refactoring.  A point brought up multiple times through the 
article is that there must be good communication in a team when constant 
refactoring is happening.  This is because a lack of communication with 
continuous refactoring will cause conflicts between sections of the project 
in question. 

\tab One way to look at AGILE strategies is that it is a development pattern whose 
organization and structure are based on its lack of a framework This makes 
it unlikely for an architecture to be made through refactoring. Refactoring 
doesn’t allow structure since it is a strategy based on in-the-moment 
development. Thapparambil stated, ”no agile methods discuss Architecture in 
any length.” A factor that can warp the architecture of a project is having 
an ineffective safe net. A safe net is set in place so that if refactoring 
goes wrong it will not crash an entire system.

\tab There are a few project-related factors mentioned in the paper, some of 
them are: the rate of change, size, type, the maturity of AK, system age. 
Each of these factors has its pros and cons when creating satisfactory 
architecture.  The factors mentioned use refactoring as their primary way of
development. 

\tab The rate of change is the speed at which refactoring occurs. Participants 
stated that a high rate of change would result in minor nuanced refactoring, 
which does not benefit a satisfactory architecture.  On the other hand, low 
refactoring rates will muddy the primary goal of the project, causing 
non-satisfactory architecture. 

\tab Participants stated how projects too small and too large would suffer from
non-satisfactory architecture.  Projects with multiple teams can have 
communication issues, while projects with few people are hard to develop a
strict architecture.  The type of project has a similar result where a 
project too small will suffer from a lack of structure, and a project too 
large could become overcomplicated. 

\tab The maturity of architectural knowledge is another prominent proponent of 
satisfactory architecture. A team will be more likely to succeed if they have 
previous architectures to reference. System age is similar to maturity where 
an older system will not have the experience needed to create a successful 
architecture.\\

\tab Since there is no set architecture in a refactoring-based development, 
recent changes are needed to be communicated. “They [the team] were used to 
work heads down in their cubicles for months without speaking to anybody. 
After that time they simply deposit hundreds of pages of useless diagrams and 
felt good about it.  [...]  the sense of responsibility that comes with Agile 
is not there” (Page 201). This quote sums up how necessary good culture and 
communication in a team is for developing satisfactory architecture through 
refactoring. 

\end{document}
